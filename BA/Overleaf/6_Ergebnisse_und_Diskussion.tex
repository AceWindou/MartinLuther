\chapter{Ergebnisse und Diskussion}

\section{Vorstellung des fertigen Videospiels}
 
\section{Diskussion der Ergebnisse und Einschätzung des Erfolgs des KI-Einsatzes}

\subsection{Einsatz von MonsterMash}
Monster Mash ist ein KI-System, mit dem man Monster erstellen kann. Wenn man sich realitätsnahe Ergebnisse wünscht, wird man mit Monster Mash auf sehr große Herausforderungen treffen.
\\
Monster sind Fantasiewesen, und niemand kann genau beschreiben, wie ein Monster aussieht. Bei der Darstellung von Menschen oder Gebäuden sieht das anders aus. Für mein Adventure-Game, mit einem historischen Hintergrund, ist MonsterMash nicht zu empfehlen.
\\
Anders würde es in einem Fantasy-Szenario aussehen, wo undefinierte Gestalten dem Spieler begegnen sollen.

\subsection{Einsatz von PFuHD}
PIFuHD ist eine KI-System was darauf trainiert ist, Digitalfotos von Personen in ein 3D-Modell umzuwandeln. PIFuHD kann man auf Google-Collab einrichten und lauffähig machen.
\\
Für das erstellen von 3D-Modellen wurde PIFuHD ist in der kostenlosen Demo-Version verwendet.
\\
Die Kompatibilität zwischen Midjourney und PIFuHD ist möglich. Die Resultate sind zum teil Artefakt belastet, die besonders in Bereichen der Hände, Füße und Kleidung auftreten.
\\
Durch Midjourney konnte ich Bilder von Martin Luther erzeugen, die als Konzeptgrafiken dienten. Diese Konzeptgrafiken habe ich PIFuHD als eingabe gegeben, und hat mir daraus folgende 3D-Modelle von Personen ausgegeben, die im Prototyp als Hauptfigur und NPCs verwendet wurden.


\section{Kritische Reflexion des Entwicklungsprozesses und Ausblick auf mögliche zukünftige Entwicklungen}