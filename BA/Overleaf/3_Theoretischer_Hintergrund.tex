\chapter{Theoretischer Hintergrund}
hallo
\section{Begriffsdefinitionen}
hallohallo
\subsection{KI-Systeme}
%https://www.bsi.bund.de/DE/Themen/Verbraucherinnen-und-Verbraucher/Informationen-und-Empfehlungen/Technologien_sicher_gestalten/Kuenstliche-Intelligenz/kuenstliche-intelligenz_node.html
Mittels maschinellen Lernens großer Datenmengen, können KI-Systeme, selbständige Lösungskompetenzen erwerben. KI-Systeme können die Fähigkeit besitzen, Eingabedaten, die nicht zu ihren Trainingsdaten vorkommen, zu verarbeiten.

\subsection{Prompt}
%https://dict.leo.org/englisch-deutsch/prompts ; https://bm-experts.de/definitionenfaq/definitionen/prompt-was-ist-das-und-wie-kann-er-eingesetzt-werden/
Aus dem Englischen, to prompt, und bedeutet so viel wie auordern oder ab fragen. Der User benutzt Prompts, um einem KI-System einen Befehl zu überreichen. Im Beispiel von ChatGPT gibt der User ein Prompt in das Chatfenster, undb ChatGPT generiert eine passende Antwort.

\subsection{NPC}
% (DKDG) Klaus Breuer - Computerspiele programmieren - Künstliche Intelligenz für Künstliche Gehirne - Kapitel 15.1 Erster Absatz Seite 113 ------ Computerspiele programmieren: künstliche Intelligenz für künstliche Gehirne / Breuer, Klaus Barcode 12155751 Rückgabe bis 12.06.2023
Non-Player Characters, kurz NPC, sind vom Computer gesteuerte Charaktere, Dorfbewohner, Tiere oder Monster. Alle Charaktere und Tiere, die sich nicht vom Spieler kontrollieren lassen. NPCs sind notwendig, um eine Spielwelt lebendig wirken zu lassen.

\subsection{Game Designer}
%Die Kunst des Game Designs : bessere Games konzipieren und entwickeln / Schell, Jesse Barcode 12486880 Kapitel 1.2 Seite 35 bis 37 DKDG
Ein Game Designer besitzt ein breites Spektrum an Fähigkeiten wie Animation, Architektur, Betriebswirtschaft, Game Engineering, Darstellende Kunst, Geschichte, Management, Mathematik, Musik, Präsentation, Soundgestaltung, Spiele und viele weitere beherrschen sollte.
Der Game Designer erschafft ein Erlebnis, wobei das Spiel nicht das Erlebnis ist, sondern nur die Möglichkeit, dem Spieler ein Erlebnis zu erleben.
%Kapitel 2.1 Seite 44

\subsection{Game Engine}
%\subsection{Design}
%Seite 49 oder eine andere gemeingültige Quelle DKDG

\subsection{Videospiel Spiel}
%Kapitel 4.2 seite 74 bis 89-- Großes Kapitel

\subsection{Verticie}
\subsection{Unwrap}
\subsection{Textur}
\subsection{Ein-Mann-Videospielentwickler}
\subsection{Dialogsystem}
\subsection{Aufforderung}
\subsection{Schlüsselwort}