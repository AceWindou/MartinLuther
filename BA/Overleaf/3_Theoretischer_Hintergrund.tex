\chapter{Theoretischer Hintergrund}
\section{Begriffsdefinitionen}

\subsection{KI-Systeme}

Unter dem Begriff “KI-Systeme” werden solche Maschinen verstanden, die in der Lage sind, eigenständig abstrakt beschriebene Aufgaben zu bearbeiten, welche nicht Schritt für Schritt vom Menschen programmiert wurden. Die KI-Systeme basieren auf maschinelles Lernen, wodurch die Systeme die Fähigkeit gewinnen weiterzulernen und vorab trainierte Modelle zu verbessern \autocite{ifaa2023}. KI-Systeme finden mittlerweile in den verschiedensten Branchen Anwendung wie der Automobilindustrie für das autonome Fahren, in der Logistik, Medizin oder in der Landwirtschaft. Im privaten Bereich sind KI-Systeme in Apps, auf dem Smartphone oder für automatisch generierte Spach- und Bilderkennung im Einsatz \autocite[S. 1, S.8]{moring2023}.

\subsection{Prompt}
%https://dict.leo.org/englisch-deutsch/prompts ; https://bm-experts.de/definitionenfaq/definitionen/prompt-was-ist-das-und-wie-kann-er-eingesetzt-werden/
Das englische Wort "prompt" leitet sich gemäß des Cambridge Dictionary \autocite{cambridge-university-press-assessment-2023} von dem Verb "to prompt" ab und bedeutet übersetzt sinngemäß, jemanden dazu zu bringen, etwas zu sagen oder zu tun. Im Bezug auf Computer bedeutet das Nomen "prompt", einem KI-System eine Anweisung zu geben, welche nicht in Computersprache verfasst wird, sondern in der natürlichen Sprache des Menschen \autocite{cambridge-university-press-assessment-2023}. 
Prompt kann demnach als eine Eingabeaufforderung verstanden werden, die in Form von beispielsweise Fragen, Aufforderungen oder auch Beschreibungen eines Themas sein kann, welche dem KI-System in natürlicher Sprache übermittelt werden, worauf das KI-System eine entsprechende Antwort generiert \autocite{BM-ExpertsGmbH2023}.

\subsection{NPC}
% (DKDG) Klaus Breuer - Computerspiele programmieren - Künstliche Intelligenz für Künstliche Gehirne - Kapitel 15.1 Erster Absatz Seite 113 ------ Computerspiele programmieren: künstliche Intelligenz für künstliche Gehirne / Breuer, Klaus Barcode 12155751 Rückgabe bis 12.06.2023
Non-Player Characters, kurz NPC, sind in Computerspielen zu finden und spielen eine wichtige Rolle, um eine Spielwelt lebendig zu gestalten \autocite[S. 293]{hack2018}. NPC sind vom Computer gesteuerte Charaktere wie Dorfbewohner, Tiere oder Monster. Alle Charaktere und Tiere, die sich nicht vom anderen Spieler kontrollieren lassen\autocite[S.113]{breuer2012}.

\subsection{Ein-Mann-Videospielentwickler}
%Die Kunst des Game Designs : bessere Games konzipieren und entwickeln / Schell, Jesse Barcode 12486880 Kapitel 1.2 Seite 35 bis 37 DKDG
Bereits in der Einleitung wurde kurz beschrieben, dass es verschiedene Wege gibt um als Videospieleentwickler zu arbeiten. Hierbei wurden die drei folgenden Wege aufgezeigt: Das Arbeiten in einem großen Videospielstudio mit sehr vielen Angestellen, das Arbeiten in einem kleineren Entwicklerstudio und der dritte Weg ist es, ein Videospiel alleine oder in einem sehr kleinen Team zu entwickeln. Wang \autocite[S.251]{wang2023} bezeichnet diesen Entwickler als solo game developer. In dieser Bachelorarbeit verwende ich den Begriff "Ein-Mann-Videospielentwickler" um zu verdeutlichen, dass alle Prozesse bei der Entwicklung meines Videospiels von nur einer einzigen Person durchgeführt werden. 

\subsection{Verticie}
\subsection{Unwrap}
\subsection{Textur}
\subsection{Dialogsystem}
\subsection{Aufforderung}
\subsection{Schlüsselwort}
\subsection{Polycount}
\subsection{Userinterface}