\chapter{Fazit}
%todo irgenwann hatte ich die schanuze voll mit chatgpt zu arbeiten, spätestens als ich versucht habe mehrer antwortmöglichkeiten zu erstellen was verschieden auswirkung in meinen prototy gehabt hätte
%todo noch einpflegen und eine eigene meineung einbinden warum ich einen zweiten ansatz gestartet habe um fachwerkhäuser zu erstellen
%todo ergebnis fuchskopf eigene meinung zufrieden usw
%Irgenwann ist mir der Kragen mit ChatGPT geplatzt was mich dazu gewungen hat 20 korrekturschleifen durchzugehen damit ich ordentliche Midjourney promts bekomme... deshalb habe ich das was ich haatte selber so geändert das es am ende passt
\begin{comment}
%%%%%%%%%%%%%%%%%%%
%%%%%%%%%%
Mit den Ergebnissen bin ich so weit zufrieden, da mein Hauptcharakter kein Abenteuerer mit Fuchskopf ist, sonder Martin Luther nachempfunden sein soll, habe ich ChatGPT dazu Aufgefordert mir 5 Promts für Midjourney zu erstellen. Die ergebnisse von ChatGPT ist in Abbildung \ref{chatgptMartinLutherMJformelErstenFünf} zu sehen. Die dazu entsandenen Bilder von Midjourney sind in Abbildung \ref{MidJourneyMLMitFormel} zu sehen.


Die erste Ansatz funktioniert. bringt aber ein Problem mit sich - die Modelle wirken etwas platt und langweilig.
Besonders da die Kamera sich direkt hinter der Spielfigur befindet und frei schwenkbar ist fallen solche langweiligen Modelle auf. Anders, wenn die Kamera so angeordnet ist, wenn sie nur von oben herab schaut. Dafür wären solche Modelle wiederum sehr gut geeignet.
\\
Dieser Ansatz ist nicht gut geeignet, denn die Hauptfigur bewegt sich in der Third-Person durch die Spielwelt. Das Problem mit der Third Person ist, dass man Objekte sehr nah betrachten kann und dadurch unschöne Modelle eher  negativ auffallen als zum Beispiel in der Top-Down-Perspektive, wo 3D-Modelle nur von oben zu betrachten sind.
\\
Mein erster Ansatz war gut, aber nicht gut genug. Ich habe mich als ein Mann Videospiel Entwicklung gegen meinen ersten Ansatz entschieden. Nach der Implementierung in der Unreal Engine 5, sahen die 3D-Modelle für mein Geschmack etwas zu plastisch und platt aus. Die Balken innerhalb. Des Fachwerks haben keine Schatten geworfen und.  Sah einfach zu künstlich aus für meinen Geschmack. Also habe ich einen zweiten Ansatz entwickelt. Der zweite Ansatz besteht aus einem Baukastensystem, jeder Balken, jede Wand und jedes Dachelement hat.  Eine eigenes Element.  Diese Elemente können kann ich die diese Elemente, die in der Regel aus. Einfachen Formen wie?
\\
Quader, oder? Dreiecke bestehen.  Kann ich einfach mit einem blueprint Skript.  Die Texturen jeweils. Hinzufügen.  Das bedeutet, ich brauche keinen Blender mehr, um ein 3 D Modell zu erzeugen, sondern mit einfachen.  Formen. Geometrien. Die über die andere Engine.   Mir zur Verfügung gestellt werden.  Mir einen einfachen Bausatz erstellen kann, die ich dann zusammengesteckt. Zusammenstecken kann. Um halt einfache Fachwerkhäuser. Zu erstellen.  Der Vorteil meines zweiten Ansatzes liegt darin, dass ich. Individuellere. Gebäude erstellen kann.
\\
Die Sicherheit. Klar, auch in Form.  Von den Nachbarnhäusern unterscheiden. 	 Ein weiterer Vorteil ist, ich kann Innenräumen gestalten, das in der ersten Version nicht. Möglich gewesen ist. Oder möglich ist.   	 Ich kann mit der zweiten Version Innenräumen auch Fußboden gestalten. Ich kann Tapeten gestalten.

\\\Dialog\\\
An dieser Stelle ist gut zu erkennen das KI-Tool sehr gute ergebnisse erzeugen kann um ein Videospielprototypen zu entwickeln, aber KI-Systeme haben ihre Grenzen, sowie ich Als Ein-Mann-Videospiele-Entwickler auch grenzen habe in der Kompetenz diese Systeme zu bedienen.
\\
Das Problem, dass ich nicht weiß wie ein Dialogsystem entwickelt wurde, und ChatGPT mir ein Falsche lösung mir überreicht hat, habe ich durch die Suchmaschiene von Google und Youtube ein Tutorial gefunden wie ich ein Dialogsystem zwischen meiner Hauptfigur und den NPCs entwickeln kann.
%%%%%%%%%%%
%%%%%%%%%%%%%%%%%%%
\end{comment}
Fragestellung: Welche Möglichkeiten bieten KI-Systeme einem Ein-Mann-Videospieler bei der Entwicklung eines Videospiels?  

In dieser Arbeit wurde der Frage nachgegangen, welche Möglichkeiten KI-Systeme einem Ein-Mann-Videospieler bei der Entwicklung eines Videospiels bieten. 

Anhand der Entwicklung meines Prototypen, habe ich herausgearbeitet, dass ….


KI-Systeme laden zum Experimentieren ein. Als Ein-Mann-Videospieler ist es möglich, einen Plot für Videospiele in Minuten zu schreiben, wofür ich Wochen benötigen würde und das nur durch einen einfachen Prompt für ChatGPT.
\\
Ich kann Konzeptgrafiken, UX-Elemente, Texturen in Hülle und Fülle mit Midjourney generieren, ohne nur einen Stift oder Grafiktablett in die Hand zu nehmen.
\\
Als Legastheniker und untalentierter Maler und Zeichner bieten mir KI-Systeme noch nie da gewesene Möglichkeiten, mich kreativ auszuleben.
\\
Die in Kapitel 5 verwendeten KI-Systeme sind nur ein Beispiel, mit denen ich gute und sinnvolle Ergebnisse für meinen Prototyp erstellt habe. Ich habe mich mit vielen weiteren KI-Systemen beschäftigt und diese ausprobiert, die aber in der Entwicklung meines Prototypes keinen Einfluss hatten.
\\
KI-Systeme wie zum Beispiel MonsterMash, mit dem man mit einem Hintergrundbild und einer Zeichnung mit der Maus ein einfaches 3D-Modell erzeugen kann. Dieses 3D-Modell kann man mit Hilfe von bewegbaren Punkten animieren.
\\
Ich habe die Photoshop-KI von Adobe ausprobiert. Mit diesem KI-System konnte ich zum Beispiel die Texturen, die ich für mein einfaches Haus von Midjourney bekomme, so bearbeiten, dass aus einer Tür ein Fenster wird.
\\
Ich wollte mit einem KI-System namens Kaedim arbeiten, da sie einer der wenigen KI-Systemen die fähigkeit besaß, aus Bildern 3D-Modelle zu erzeugen. Der Preis von circa 15€ pro Prompt hat mich abgeschreckt. Hinzu kam das Kaedim eine sehr lange bearbeitungszeit benötigt hätte, von ca 20 Minuten war die Rede, was ich in diesem Moment sehr Merkwürdig fand.
Durch ein Blick in der Dokumentation, wurde Kaedim so beschrieben, das es eine Mischung aus einem Trainierten KI-System und 3D-Artist ist, von dem ich das Ergebnis bekommen würde.
Da ich mich als Ein-Mann-Videospielerentwickler sehe, habe ich mich dazu entschieden, dieses KI-System zu verwenden, da ich erstmal keinen anderen Menschen an meinem Prototyp arbeiten lassen möchte.
\\
Ich habe mich auch auf Soundful Music und AIVA angemeldet und versucht in Richtung Music zu forschen und Experimentieren, aber meine Zeit das leider nicht zugelassen.
\\
Musik finde ich ein Spannendes gebiet, und es gibt KI-Systeme die sich in diesem Thema sich bewegen. Leider hatte ich keine zeit im Rahmen meiner Bachelorthesis mich in dieser Richtung intensiv zu experimentieren.
\\
Während meiner Projektphase hatte ich immerwieder das gefühl, das das Wort KI und AI oft als Marketing buzzwort benutz haben um sich vom Markt abzuheben.
\\
KI-Systeme zu erforschen und erfahren, kann sehr viel zeit in Anspruch nehmen. Mit jedem Neuen System muss man üben. Jedes KI-System hat grenzen die die Erwartungen oder Anforderung eines Ein-Mann-Videospieleentwickler entsprechen.
\\
Ich möchte in diesem Fazit ein wenig bildlich werden.
\\
Wenn ein Videospiel Prototyp ein dreistöckiges Haus wäre:
Das Fundament, der Keller und alle Leitungen wie Wasser, Strom und Gas ist die Unreal Engine 5.
Der Erste- sowie die Hälfte des Zweiten-Stocks, der größte Teile der Außen- und Innenfassade, können Ergebnisse von KI-Systemen sein.
Der Rest hängt von dem Ein-Mann-Videospielentwickler und seinen Fähigkeiten ab um ein schlüsselfertiges Haus zu bauen.
\\
Mein Prototyp hat eine erkennbare Form, ist aber noch weit davon entfernt um Potenzielle Geldgeber, wie die Deutsche Gameförderung, zu überzeugen um mich als Ein-Mann-Videospielentwickler Geld zu bekommen um aus meinem Prototyp ein Fertiges, marktreifes Spiel zu erschaffen.
\\
Mein Ziel war es, Inhalte für mein Prototyp mit Hilfe von KI-Systemen für die Unreal Engine 5 zu erschaffen, und ich bin So


\section{Zusammenfassung der Ergebnisse}
\section{Implikationen für die Praxis}
\section{Limitationen der Studie}