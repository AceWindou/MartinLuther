\chapter{Fazit}

Fragestellung: Welche Möglichkeiten bieten KI-Systeme einem Ein-Mann-Videospieler bei der Entwicklung eines Videospiels?  

In dieser Arbeit wurde der Frage nachgegangen, welche Möglichkeiten KI-Systeme einem Ein-Mann-Videospieler bei der Entwicklung eines Videospiels bieten. 

Anhand der Entwicklung meines Prototypen, habe ich herausgearbeitet, dass ….


KI-Systeme laden zum Experimentieren ein. Als Ein-Mann-Videospieler ist es möglich, einen Plot für Videospiele in Minuten zu schreiben, wofür ich Wochen benötigen würde und das nur durch einen einfachen Prompt für ChatGPT.
\\
Ich kann Konzeptgrafiken, UX-Elemente, Texturen in Hülle und Fülle mit Midjourney generieren, ohne nur einen Stift oder Grafiktablett in die Hand zu nehmen.
\\
Als Legastheniker und untalentierter Maler und Zeichner bieten mir KI-Systeme noch nie da gewesene Möglichkeiten, mich kreativ auszuleben.
\\
Die in Kapitel 5 verwendeten KI-Systeme sind nur ein Beispiel, mit denen ich gute und sinnvolle Ergebnisse für meinen Prototyp erstellt habe. Ich habe mich mit vielen weiteren KI-Systemen beschäftigt und diese ausprobiert, die aber in der Entwicklung meines Prototypes keinen Einfluss hatten.
\\
KI-Systeme wie zum Beispiel MonsterMash, mit dem man mit einem Hintergrundbild und einer Zeichnung mit der Maus ein einfaches 3D-Modell erzeugen kann. Dieses 3D-Modell kann man mit Hilfe von bewegbaren Punkten animieren.
\\
Ich habe die Photoshop-KI von Adobe ausprobiert. Mit diesem KI-System konnte ich zum Beispiel die Texturen, die ich für mein einfaches Haus von Midjourney bekomme, so bearbeiten, dass aus einer Tür ein Fenster wird.
\\
Ich wollte mit einem KI-System namens Kaedim arbeiten, da sie einer der wenigen KI-Systemen die fähigkeit besaß, aus Bildern 3D-Modelle zu erzeugen. Der Preis von circa 15€ pro Prompt hat mich abgeschreckt. Hinzu kam das Kaedim eine sehr lange bearbeitungszeit benötigt hätte, von ca 20 Minuten war die Rede, was ich in diesem Moment sehr Merkwürdig fand.
Durch ein Blick in der Dokumentation, wurde Kaedim so beschrieben, das es eine Mischung aus einem Trainierten KI-System und 3D-Artist ist, von dem ich das Ergebnis bekommen würde.
Da ich mich als Ein-Mann-Videospielerentwickler sehe, habe ich mich dazu entschieden, dieses KI-System zu verwenden, da ich erstmal keinen anderen Menschen an meinem Prototyp arbeiten lassen möchte.
\\
Ich habe mich auch auf Soundful Music und AIVA angemeldet und versucht in Richtung Music zu forschen und Experimentieren, aber meine Zeit das leider nicht zugelassen.
\\
Musik finde ich ein Spannendes gebiet, und es gibt KI-Systeme die sich in diesem Thema sich bewegen. Leider hatte ich keine zeit im Rahmen meiner Bachelorthesis mich in dieser Richtung intensiv zu experimentieren.
\\
Während meiner Projektphase hatte ich immerwieder das gefühl, das das Wort KI und AI oft als Marketing buzzwort benutz haben um sich vom Markt abzuheben.
\\
KI-Systeme zu erforschen und erfahren, kann sehr viel zeit in Anspruch nehmen. Mit jedem Neuen System muss man üben. Jedes KI-System hat grenzen die die Erwartungen oder Anforderung eines Ein-Mann-Videospieleentwickler entsprechen.
\\
Ich möchte in diesem Fazit ein wenig bildlich werden.
\\
Wenn ein Videospiel Prototyp ein dreistöckiges Haus wäre:
Das Fundament, der Keller und alle Leitungen wie Wasser, Strom und Gas ist die Unreal Engine 5.
Der Erste- sowie die Hälfte des Zweiten-Stocks, der größte Teile der Außen- und Innenfassade, können Ergebnisse von KI-Systemen sein.
Der Rest hängt von dem Ein-Mann-Videospielentwickler und seinen Fähigkeiten ab um ein schlüsselfertiges Haus zu bauen.
\\
Mein Prototyp hat eine erkennbare Form, ist aber noch weit davon entfernt um Potenzielle Geldgeber, wie die Deutsche Gameförderung, zu überzeugen um mich als Ein-Mann-Videospielentwickler Geld zu bekommen um aus meinem Prototyp ein Fertiges, marktreifes Spiel zu erschaffen.
\\
Mein Ziel war es, Inhalte für mein Prototyp mit Hilfe von KI-Systemen für die Unreal Engine 5 zu erschaffen, und ich bin So


\section{Zusammenfassung der Ergebnisse}
\section{Implikationen für die Praxis}
\section{Limitationen der Studie}