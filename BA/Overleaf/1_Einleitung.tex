\chapter{Einleitung}

"'Sprechen Sie einfach diese magischen Worte aus: Ich bin ein Game Designer. (...) Haben Sie es gemacht? Wenn ja, dann gratuliere ich Ihnen. Sie sind jetzt ein Game Designer."\cite[S. 5-6]{schell2020}
\\
Dieses Zitat stammt von Jesse Schell; Hochschullehrer für Unterhaltungstechnologie am Entertainment Technology Center in Pittsburgh, USA.
\\
Wenn er gefragt wird, was er macht, um seine Brötchen zu verdienen, antwortet er: "'Ich bin Game Designer". Jesse Schell, ermutigt Anfänger in seinem Buch, die noch vor Ihrem ersten Schritt Game Designer oder Videospiele Entwickler stehen, sich selbst als Gamedesigner zu bezeichnen. Wenn wir den Worten von Jesse Schell Glauben schenken, ist Gamedesigner werden nicht schwer.
\\
Wenn man sich dazu entscheidet Videospiele zu entwickeln, stehen oft nur drei Wege zur Verfügung. Der erste Weg ist es in einem Großen Videospielentwicklungsstudium zu arbeiten die mehrere Hundert Angestellte haben. In diesen Studios ist es üblich mit einem sehr kleinen Aufgabenbereich beschäftigt zu sein. Der zweite Weg ist in einem kleinerem Entwicklerstudio anzufangen wo sich die Rollen in der Videospielentwicklung mit anderen aus dem Team Überschneiden. Der Vorteil in diesen beiden Wegen ist, das ein Entwickler von anderen Erfahreren Entwickler lernen kann. Der dritte und letzte Weg ist der Weg als Ein-Mann-Videospielentwickler. Als Ein-Mann-Videospielentwickler ist man gezwungen alle Aufgaben zu übernehmen die Anfallen um ein Videospiel zu entwickeln.
\\
In dieser Arbeit möchte ich mich mit der Frage auseinandersetzen wie KI-Systeme in der Entwicklung eines Videospiels unterstützen kann. Ich werde diese Frage als Ein-Mann-Videospielentwickler nachgehen und im Rahmen dieser Bachelorthesis ein Prototyp entwickeln und in meinem Projekt so viele KI-Systeme einbinden wie es Möglich und Sinnvoll ist.

%https://dailygame.at/baldurs-gate-3-entwickler-bleibt-weiterhin-unabhaengig/ -03.09.2023
 
\section{Motivation und Idee}
Länger als ich denken kann, spiele ich Videospiele. Das Nintendo Entertainment System gehörte zu meiner Welt als Kind wie mein zu Hause und die Natur draußen. Obwohl wir nie viel Geld hatten, hatten wir doch an dem kleinen Rhörenfernsehr im Wohnzimmer diesen Wunderkasten in dem ich in Super Mario die Prizessen retten musste, Enten mit dem NES Zapper in Duck Hunt jagte, und in Teenage Mutant Ninja Turtels 2 jeden Sieg mit dem Schlachtruf Cowabunga feierte.
\\
Während ich als Kind nie das letzte Level in Teenage Mutant Ninja Turtel geschafft habe, war ich so frustriert das ich gesagt habe, das das Spiel niemand schaffen kann. Mein Bruder Patrick hat mich in diesem moment getröstet und gesagt: "Doch, es gibt einen der das Spiel schafft, der Entwickler!"
\\
Das war der Moment, wo ich verstand, dass hinter jedem Werk auch ein Entwickler stand, und so ein Mensch wollte ich immer werden. Doch ich bin auf einem Dorf mit 300 Einwohnern aufgewachsen und hatte nie die Chance Medienkompetenzen zu erlangen, mit denen ich mir eine Perspektive aufbauen konnte um Videospielentwickler zu werden.
\\
Während meines Studium an der Hochschule Fulda beschäftige ich mich immer mehr mit dem Entwickeln von Videospielen. Obwohl ich ein paar Gruppenarbeiten im Rahmen meines Studiums gemeistert habe, bemerkte ich, dass das Thema Videospiele zu entwickeln bei meinem Kommilitonen nie das große Thema war. Also schlug ich den weg als Ein-Mann-Videospielentwickler ein.
\\
Auf meinem Weg als Ein-Mann-Videospielentwickler habe ich bemerkt, dass ich gute Programmiertkäntnisse besitze, und auch mit Klängen gut umgehen kann, aber wenn ich ein Stift in die Hand nehme um was zu schreiben oder zu malen, habe ich immer gemerkt, dass andere immer schneller und besser sind.
\\
Dieser Gedanke, verlor ich im Oktober und Dezember 2022, denn seit dieser Zeit experimentiere ich mit KI-Systemen, die mich dabei unterstützen in diesen Disziplinen kreativ tätig zu sein.
\\
Videospielentwicklung und KI-Systeme sind zwei Welten. Und diese zwei Welten möchte ich mit einer Brücke verbinden.
\\
\begin{comment}
	
Durch mein Studium in Digitale Medien bin ich erst dazu gekommen mich mit Gameengine

Das war der Moment, wo ich begriff, das hinter jedem dieser Werke jemand Steckt der so was Entickelt.
\\
Spielen war schon schwer, aber Entwickeln? - Unmöglich.
Videospiele werden aus sehr vielen Teilbereichen der Medienbranche zusammengesetzt, wie zum Beispiel Autoren, Programmierer und Illustratoren bis hin zu Marketing und Vertrieb.
\\
Die Vergangenheit hat merfach bewiesen das Videospiele von einer Person entwickelt werden könne. Spiele wie Stardew Valley das von Eric Barone entwickelt wurde, Minecraft das ursprünglich von Markus „Notch“ Persson entwickelt wurde oder Undertale das von Toby Fox entwickelt wurde.
\\

Spätestens in den 90er Jahren wurden nur noch sehr wenige Spiele von einer Person entwickelt die in Arcadehallen oder Kaufhäußer zugänglich waren.
Warum der Ein-Mann-Videospieleentwickler immer seltener wurde, liegt großteil daran, dass die Technik auf denen die Vidoepiele liefen mit der Zeit leistungsfähiger wurden und somit größere und Komplexere Spielewelten erschaffen konnten.
\\
Diese Komplexität der Spielewelten konnte nicht mehr von einem einzelnen Entwickler gewährleistet werden.
\\

Im Jahr 2022 hat die Firma OpenAI sein KI-Werkzeug ChatGPT der Öffentlichkeit zugänglich gemacht, und viele Berichte über einen Meilenstein in der KI-Forschung.
\\
ChatGPT kann selbständig durch eine für den Menschen einfache Prompt Schulaufgaben lösen oder sich mit dem Benutzer unterhalten.
\\
ChatGPT kann ganze Programme in verschiedenen Programmiersprachen Schreiben, was es davor nie in solchen Umfang da gewesen war.
\\
KI-Systeme bieten ein neues Gebiet um Forschung und Experimente zu betreiben, und ich möchte in meiner Bachelorthesis herausfinden ob es möglich ist, ein Videospiel mit hilfe von KI-Systemen zu entwickeln, so wie in der Pionierzeit wo einzelne Entwickler ganze Projekte erschaffen haben.
\\ 
Die Systeme, auf denen Videospiele liefen, wurden immer leistungsfähiger, und somit wurden auch lebendigere und komplexere Welten möglich. Videospiele wurden in der Regel nicht mehr von einer Person entwickelt, sondern von ganzen Studios. In diesen Studios werden Aufgaben auf Teams verteilt, wie zum Beispiel Concept Art and Design, Musik und Soundeffekte bis hin zum Vertrieb und Marketing.
\\
Kurz, ein Videospiel zu entwickeln ist schon sehr lange keine Ein-Mann-Aufgabe mehr, Und in solchen Teams kann jeder Videospielentwickler sich auf seine Stärken im Team konzentrieren.
\\
Ich sehe seit 2022 eine neue Möglichkeit Videospiele zu entwickeln, die zuvor in diesem Umfang nicht möglich gewesen war.
\\
KI-Systeme sind Werkzeuge, die ein hohes Potenzial beinhalten, um schnelles und qualitatives Arbeiten mit sich bringen.
\\
Mit Midjourney kann ich innerhalb von wenigen Minuten eine Landschaft erstellen lassen. ChatGPT kann dir Geschichten schreiben und Voice.ai dir eine neue Stimme verleihen. Das was die vorhin drei genannten KI-Systeme sich spezialisiert haben, sind in der realen
Welt, echte Berufe in der Gamingbranche - Concept Artist, narrative Designer / video game writer, voice actor.
\\
Es ist heute theoretisch möglich, ohne viele Vorkenntnisse diese Aufgaben mit Hilfe von KI-Systemen zu übernehmen.
Inhalt...
\end{comment}
\section{Forschungsfrage und Forschungsmethode}
Am ende Dieser Bachelorthesis möchte ich die Frage beantworten Wie KI-Systeme Videospielentwickler unterstützen können um ein Videospiel zu entwickeln.
Weitere Fragen die ich in meiner Bachelorthesis beantworten möchten sind die Hürden und Grenzen von KI-Systemen in der Videospielentwicklung und welche Vorteil und Nachteile bringen sie mit der benutzung und ist es sinnvoll auf solche KI-Systeme zurückzugreifen?
Diese Fragen gehe ich als Ein-Mann-Videospielentickler nach, indem ich ein Prototyp mit Hilfe von KI-Systeme entwickel.
\section{Gliederung der Arbeit}%todo hööööö?!
Die Arbeit ist gegliedert in, Warum ich was mache, dann ein Paar allgemeine Erklärungen.
\\
Was ich mache. Und welche Werkzeuge ich benutze. Ich werde Alles am Ende analysieren

\section{Zielsetzung}
Mein Ziel in dieser Arbeit ist es, einen Prototyp zu entwickeln. Dieser Prototyp wird nur von einer Peron entwickelt, und alle anderen Aufgaben und Probleme werden versucht, mit Hilfe von KI-Systemen zu lösen.

\section{Abgrenzung}
